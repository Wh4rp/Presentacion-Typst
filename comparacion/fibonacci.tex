\documentclass{article}
\usepackage{booktabs}
\usepackage{amsmath}
\usepackage{geometry}

\geometry{a4paper, margin=2cm}

\pagestyle{empty}

\begin{document}

\begin{center}
  {\LARGE \textbf{Secuencia de Fibonacci}}
\end{center}

\section{Definición Recursiva}

La secuencia de Fibonacci se define de forma recursiva como:

\begin{equation*}
    F_n = \begin{cases}
               0 & \text{si } n = 0 \\
               1 & \text{si } n = 1 \\
               F_{n-1} + F_{n-2} & \text{si } n > 1
           \end{cases}
\end{equation*}

\section{Ecuación Cerrada}

La secuencia de Fibonacci también se puede expresar mediante la siguiente ecuación cerrada:

\begin{equation*}
    F_n = \frac{1}{\sqrt{5}}\left[\left(\frac{1 + \sqrt{5}}{2}\right)^n - \left(\frac{1 - \sqrt{5}}{2}\right)^n\right]
\end{equation*}

\section{Tabla de Valores}

A continuación se muestra una tabla que presenta los primeros 10 números de la secuencia de Fibonacci:

\begin{table}[h]
\centering
\begin{tabular}{@{}ll@{}}
\toprule
$n$ & $F_n$ \\ \midrule
0 & 0 \\
1 & 1 \\
2 & 1 \\
3 & 2 \\
4 & 3 \\
5 & 5 \\
6 & 8 \\
7 & 13 \\
8 & 21 \\
9 & 34 \\ \bottomrule
\end{tabular}
\caption{Los primeros 10 números de la secuencia de Fibonacci.}
\end{table}

\end{document}
